%%%%%%%%%%%%%%%%%%%%%%%%%%%%%%%%%%%%%%%%%
% Important note:
% This template requires the resume.cls file to be in the same directory as the
% .tex file. The resume.cls file provides the resume style used for structuring the
% document.
%
%%%%%%%%%%%%%%%%%%%%%%%%%%%%%%%%%%%%%%%%%

%----------------------------------------------------------------------------------------
%	PACKAGES AND OTHER DOCUMENT CONFIGURATIONS
%----------------------------------------------------------------------------------------

\documentclass{resume} % Use the custom resume.cls style

\usepackage[left=0.75in,top=0.6in,right=0.75in,bottom=0.6in]{geometry} % Document margins
\usepackage[colorlinks=true, urlcolor=blue, linkcolor=red]{hyperref}
\newcommand{\tab}[1]{\hspace{.2667\textwidth}\rlap{#1}}
\newcommand{\itab}[1]{\hspace{0em}\rlap{#1}}
\name{debjit mandal} % Your name
\address{\href{https://debjit-mandal.com}{{https://debjit-mandal.com}}} % Your address
%\address{123 Pleasant Lane \\ City, State 12345} % Your secondary addess (optional)
\address{(+91) 8335024039 \\ debjitmandal8925@gmail.com} % Your phone number and email

\begin{document}

%----------------------------------------------------------------------------------------
%	EDUCATION SECTION
%----------------------------------------------------------------------------------------

\begin{rSection}{Education}
%--copy and paste this region  if you need more--
{\bf Kalinga Institute of Industrial Technology} \hfill {\em 2022 - 2026} 
\\ B.Tech, Computer Science and Engineering \hfill { GPA: 9.15 }

\\
%--copy and paste this region  if you need more--

\end{rSection}
%----------------------------------------------------------------------------------------
%	EXPERIENCE SECTION
%----------------------------------------------------------------------------------------
\begin{rSection}{Experience}
%--copy and paste this region  if you need more--
\href{https://1drv.ms/i/s!AqmF3ryI8xyngdlvi5puoHlWRF5NTQ?e=hyfCRH}{{\bf CAMPUS AMBASSADOR}{, IMUNA}} \hfill {\em JULY, 2023- AUGUST, 2023}
\begin{itemize}
    \item Promoted IMUN Online Conferences in different schools/universities and got 10+ registrations.
    \item Best Intern for a week.
\end{itemize}
%--copy and paste this region  if you need more--

\end{rSection}
%--------------------------------------------------------------------------------
%    PROJECTS
%-----------------------------------------------------------------------------------------------
\begin{rSection}{Projects}
%--copy and paste this region  if you need more--
\begin{itemize}
\item{\href{https://github.com/debjit-mandal/dsh}{{\bf DSH:}}
 A Command-Line shell made using Python.
\end{itemize}
%--copy and paste this region  if you need more--
%--copy and paste this region  if you need more--
\begin{itemize}
\item{\href{https://github.com/debjit-mandal/movielens-data-analysis}{{\bf Movielens Data Analysis:}}
 Data analysis of Movielens 1M dataset.
\end{itemize}
%--copy and paste this region  if you need more--
%--copy and paste this region  if you need more--
\begin{itemize}
\item {\href{https://github.com/debjit-mandal/steganography}
{{\bf Steganography:}}
 Text-based steganography using Python.
\end{itemize}
%--copy and paste this region  if you need more--
%--copy and paste this region  if you need more--
\begin{itemize}
\item{\href{https://github.com/debjit-mandal/code-plagiarism-checker}{{\bf Code Plagiarism Checker:}}
 A code plagiarism checker made using Python.
\end{itemize}
%--copy and paste this region  if you need more--
%--copy and paste this region  if you need more--
\begin{itemize}
\item{\href{https://github.com/debjit-mandal/jira-fetcher}{{\bf Jira Fetcher:}}
 Jira Data Extractor using Python.
\end{itemize}
%--copy and paste this region  if you need more--
%--copy and paste this region  if you need more--
\begin{itemize}
\item{\href{https://github.com/debjit-mandal/reddit-scraper}{{\bf Reddit Scraper:}}
 Reddit Scraper made using Python.
\end{itemize}
%--copy and paste this region  if you need more--
%--copy and paste this region  if you need more--
\begin{itemize}
\item{\href{https://github.com/debjit-mandal/gmail-fetcher}{{\bf Gmail Fetcher:}}
 Emails Fetcher from a particular Gmail user using Python And Gmail API.
\end{itemize}
%--copy and paste this region  if you need more--
\end{rSection}
%--------------------------------------------------------------------------------
%----------------------------------------------------------------------------------------
%	SKILLS SECTION
%----------------------------------------------------------------------------------------
\begin{rSection}{PROGRAMMING LANGUAGES AND TOOLS}

\\Python, scikit-learn, NumPy, pandas, TensorFlow, Excel, PowerBI, MySQL, Matplotlib, C, NLP, JavaScript, Go, Rust, MongoDB, Git, SourceHut, GitHub, GitLab\\\\
\end{rSection}
\end{document}----------------------------

